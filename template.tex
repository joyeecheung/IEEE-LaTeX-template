%--------------------------------------------------------------------------------------------------------------------------------------------------
%\documentclass[letterpaper, 10 pt, conference]{ieeeconf}  % Comment this line out
                                                           % if you need a4paper
\documentclass[a4paper, 10pt, conference]{ieeeconf}        % Use this line for a4
                                                           % paper
%--------------------------------------------------------------------------------------------------------------------------------------------------

\usepackage{lipsum}
%---------------------------------------------- LOCALLY USED PACKAGES ------------------------------------------------------%
% More packages can be found on http:\\www.ctan.org. They can be automatically be downloaded using MikTek's Package Manager.
%---------Figure Packages-------------------------------
\usepackage{graphicx}                       % For figures
\usepackage{stfloats}
\usepackage[tight,footnotesize]{subfigure}  % Create subfigures, ie 1A, 1B
%\usepackage{epsfig} % for postscript graphics files

%---------Math Packages---------------------------------
\usepackage{amssymb,amsmath}
%\usepackage{amslatex}%
\usepackage{textcomp}
\usepackage{mdwmath}
\usepackage{mdwtab}
\usepackage{eqparbox}
%---------Table Packages--------------------------------
\usepackage{rotating}                       % Used to rotate tables
\usepackage{array}                          % Fixed column widths for tables
%---------Algorithm Packages----------------------------
\usepackage{listings}                       % Source code
\usepackage{algorithm}                      % Pseudo Code
\usepackage[noend]{algpseudocode}

\usepackage{hyperref}

% New Commands
\setlength{\parskip}{1pt}                   % Changes the space between paragraphs.
\newcommand{\squeezeup}{\vspace{-2.5mm}}    % changes space after figures. Use \squeezup to use.
%------------------------------------------------------------------------------------------------------------------
%\renewcommand{\arraystretch}{0.95} % Table Row Spacing. Changes height of tables.
\setlength{\parskip}{1pt}           % Changes the space between paragraphs.
%\parindent - the size of the paragraph indent
%\baselineskip - vertical distance between lines.
%\parskip - the extra space between paragraphs.
%\textwidth - the width of a line of text in the local environment (e.g., the lines are commonly narrower in the abstract than in the normal text).
%\textheight - the height of the text on the page.
%--------------------------------------------------------------------------------------------------------------------------------------------------
\title{\LARGE \bf Title Here.}
\author{ \parbox{3 in}{ \centering Juan Rojas*
          Vision and Manipulation Group\\
          Intelligent Systems Research Institute\\
          National Institute of Advanced Industrial Science and Technology\\
          Tsukuba, Ibaraki, 305-8568, Japan
          {\tt rojas@aist.go.jp}}
          \hspace*{ 0.5 in}
	      \parbox{3 in}{ \centering Kensuke Harada
	       Vision and Manipulation Group\\
	       Intelligent Systems Research Institute\\
	       National Institute of Advanced Industrial Science and Technology\\
	       Tsukuba, Ibaraki, 305-8568, Japan\\
	      {\tt kensuke.harada@aist.go.jp}}
} 
\begin{document}

\maketitle
\thispagestyle{empty}
\pagestyle{empty}

%--------------------------------------------------------------------------------------------------------------------------------------------------
\begin{abstract}
%--------------------------------------------------------------------------------------------------------------------------------------------------
% If useful, and where word limit allows, include:
%   - One or two sentences of background information (placed at the beginning)
% Statement of:
%   - The question asked (present verb tense)
%   - What was done to answer the question (past verb tense) - research approach, population studies, independent and dependent variables
%   - Findings that answer the question (past verb tense) - the most important results and evidence (data) presented in a logical order.
%   - The answer to the question (present verb tense)
% An implication or a speculation based on the answer (present verb tense, placed at the end)
%--------------------------------------------------------------------------------------------------------------------------------------------------

\lipsum[2-3]
\end{abstract}
%--------------------------------------------------------------------------------------------------------------------------------------------------

%--------------------------------------------------------------------------------------------------------------------------------------------------
\section{INTRODUCTION}
%--------------------------------------------------------------------------------------------------------------------------------------------------
% Background to the topic (past verb tense)
%   -	What is known or believed about the topic
%   -	What is still unknown or problematic
%   -	Findings of relevant studies (past verb tense)
%   -	Importance of the topic
%
% Statement of the research question
%   -	Several ways can be used to signal the research question , e.g.,
%   -	"To determine whether ¡­¡­¡­"
%   -	"The purpose of this study was to ¡­¡­."
%   -	This study tested the hypothesis that ¡­¡­"
%   -	"This study was undertaken to ¡­¡­"
%
% Approach taken to answer the question (past verb tense)
% Results and answer to the way in which the problem statement was addressed (present tense)
%--------------------------------------------------------------------------------------------------------------------------------------------------
\section{Methods}
%--------------------------------------------------------------------------------------------------------------------------------------------------
\lipsum[3]\cite{1990ICRA-Simon-SelfTuningRobotPrimitivesSnapFit}
% Intro

%--------------------------------------------------------------------------------------------------------------------------------------------------
\subsection{Sub Method}
%--------------------------------------------------------------------------------------------------------------------------------------------------
\lipsum[2-3]
%--------------------------------------------------------------------------------------------------------------------------------------------------
\subsubsection{Subsub Method}
%--------------------------------------------------------------------------------------------------------------------------------------------------

\lipsum[3]
%--------------------------------------------------------------------------------------------------------------------------------------------------
\section{Experiments}\label{sec:Experiments}
%--------------------------------------------------------------------------------------------------------------------------------------------------
% General Introduction

%--------------------------------------------------------------------------------------------------------------------------------------------------
\subsubsection{Testbed Description}
%--------------------------------------------------------------------------------------------------------------------------------------------------
\lipsum[2-3]
%--------------------------------------------------------------------------------------------------------------------------------------------------
\subsubsection{Experimental Details} \label{subsec:Experimental Details}
%--------------------------------------------------------------------------------------------------------------------------------------------------

%--------------------------------------------------------------------------------------------------------------------------------------------------
\subsubsection{Performance Metrics} \label{subsec:Performance Metrics}
%--------------------------------------------------------------------------------------------------------------------------------------------------


%--------------------------------------------------------------------------------------------------------------------------------------------------
\subsection{Experiment 1: HP3JC-Push, ISAC-Hold}\label{sec:Exp1}
%--------------------------------------------------------------------------------------------------------------------------------------------------
% Intro to the experiment

%--------------------------------------------------------------------------------------------------------------------------------------------------
\subsection{Results}\label{subsec:Results}
%--------------------------------------------------------------------------------------------------------------------------------------------------
% Intro to results
%--------------------------------------------------------------------------------------------------------------------------------------------------
\subsubsection{Result 1}
%--------------------------------------------------------------------------------------------------------------------------------------------------

%--------------------------------------------------------------------------------------------------------------------------------------------------
\section{DISCUSSION} \label{sec:Discussion}
%--------------------------------------------------------------------------------------------------------------------------------------------------
%1.	Answers to the question(s) posed in the introduction together with any accompanying support, explanation and defense of the answers (present verb tense) with reference to published literature. % Summarize your evidence for each conclusion (i.e. don't assume anything).
%2.	Explanations of any results that do not support the answers.
%3.	Indication of the originality/uniqueness of the work
%4.	Explanations of:
%-	How the findings concur with those of others
%-	Any discrepancies of the results with those of others
%-	Unexpected findings
%-	The limitations of the study which may affect the study validity or generalisability of the study findings.
%5.	Indication of the importance of the work e.g. clinical significance
%6.	Recommendations for further research
%--------------------------------------------------------------------------------------------------------------------------------------------------
\section{CONCLUSION} \label{sec:Conclusion}

\lipsum[2-3]\cite{1982AI-Brooks-SymbolicErrorAnalysisBotPlanning}
%--------------------------------------------------------------------------------------------------------------------------------------------------
% Concluding Thoughts
% Brief statement of the major findings and implications of the study
%------------------------------------------------------------------------------------------------------------------
%% BIBLIOGRAPHY
%------------------------------------------------------------------------------------------------------------------
\bibliographystyle{IEEEtran}
\bibliography{xbib}
\end{document}